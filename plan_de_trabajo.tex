\documentclass{article}
\usepackage[spanish]{babel}
\usepackage[utf8]{inputenc}

\title{Plan de trabajo}
\author{Fabian Selles Rosa \\ \emph{fabian.sellesrosa@alum.uca.es}}
\date{}
\begin{document}
\maketitle

\section{Lineas Generales:}
\begin{enumerate}
\item Integración de SER como SIP router con una PBX por debajo sea cual sea la utilizada.
\item Implantación de click2dial con dos posibilidades
  \begin{enumerate}
  \item Con un softphone integrado, se puede hablar desde el ordenador directamente.
  \item Con un teléfono fijo, se conecta la llamada desde la extensión a la que se quiere llamar con un numero introducido por el usuario.
  \end{enumerate}

\item Implantación de servidores Jabber(XMPP)

\item Implantación de IVR.
\end{enumerate}

\section{Lineas concretas de trabajo.}
\label{sec:lineas-concretas-de}

\begin{itemize}
\item Instalación de un sistema Asterisk que haga de \emph{gateway\footnote{pasarela o convertidor de medios.}} entre la centralita \emph{SipX} y la centralita PBX de telefonía tradicional. Permitiendo integrar la centralita de pruebas \emph{SipX} con la telefonía tradicional. Pasos necesarios
  \begin{enumerate}
  \item Elaboración de un presupuesto.
  \item Instalación del \emph{Gateway}
  \item Pruebas y depuración.
  \end{enumerate}
\end{itemize}

\section{Posibilidades a trabajar}
\label{sec:posib-trab}

\begin{itemize}
\item Para el epigrafe 1:
  \begin{itemize}
  \item Tener servidores redundantes, si un servidor cae, la reserva se levanta mediante heartbeat.
  \item Tener servidores redundantes con balanceo de carga.
  \item Implementar una solución de facturación, bien de desarrollo propio o mediante herramientas externas como CDRtool.
  \end{itemize}
\item Para el epigrafe 2:
  \begin{itemize}
  \item Click2dial en web corporativas a servicios corporativos, profesores, etc.
  \item Servicios corporativos a través de la mensajería instantanea.
  \end{itemize}
\item Para el epigrafe 3:
  \begin{itemize}
  \item Mensajeria instantanea interna.
  \item Utilizar la mensajeria instantanea para comentar en foros de Moodle por ejemplo.
  \item O para comentar en una página web de la UCA.

  \item Integración con VoIP para la presencia y la mensajería desde el cliente VoIP.

  \item Salas de charla, por ejemplo, por curso de Moodle o por titulación, produciendo un espacio de intercambio común de opinión.

  \item Posibilidad de "transportar" mensajes XMPP a una página web, un ejemplo, profesor se reune con alumno y se deja constancia en una página web que otros pueden consultar.

  \item PAS que recibe preguntas mediante mensajeria instantanea, se guardan las conversaciones para poder mejorar.

  \end{itemize}

\item Para el epigrafe 4:
  \begin{itemize}
  \item Mediante Asterisk se pueden crear y gestionar IVR con bastantes capacidades, un ejemplo, gestión de notas de selectividad.

  
  \end{itemize}
\end{itemize}
\end{document}

%%% Local Variables: 
%%% mode: latex
%%% TeX-master: t
%%% End: 
