\chapter{Anexos}

\section{Codigo fuente}

\subsection{\emph{container\_recycler}}
\label{subsec:containe-recycler-src-code}

Se incluye un extracto del código de \emph{container\_recycler} por completitud y como soporte a la lectura de la memoria. Sin
embargo se recomienda su exploración en GitHub en \href{https://github.com/Fsero/container\_recycler}{https://github.com/Fsero/container\_recycler} , dónde será más fácil interactuar con el código y ver cambios recientes si los 
hubiera.

\begin{minted}[fontsize=\scriptsize]{go}
    package main
    
    import (
        "bitbucket.org/fseros/container_recycler/handlers"
        "bufio"
        "context"
        "os"
        "runtime"
        "strings"
    )
    
    func main() {
        runtime.GOMAXPROCS(runtime.NumCPU())
        handlers.SetupLogging()
        ctx, cancel := context.WithCancel(context.Background())
        defer cancel()
    
        // TODO: Replace with config file
        ctx = context.WithValue(ctx, "exposure_time", "10m")
        ctx = context.WithValue(ctx, "container_api_timeout", "10s")
        ctx = context.WithValue(ctx, "tmp_flags_file_path", "/var/tmp/container_recycler_")
    
        // reading arguments
    
        for _, arg := range os.Args[1:] {
            r := strings.NewReader(arg)
            handlers.ParseFalcoNotifications(r, ctx)
        }
    
        // read from stdin
        r := bufio.NewReader(os.Stdin)
        handlers.ParseFalcoNotifications(r, ctx)
    
    }
    
\end{minted}
\captionof{listing}{container\_recycler - main.go \label{listing:container-recycler-main}}

\begin{minted}[fontsize=\scriptsize]{go}
    package handlers
    
    import (
        "context"
        "fmt"
        "io/ioutil"
        "os"
        "runtime"
        "strings"
        "time"
    
        log "github.com/Sirupsen/logrus"
        "github.com/docker/docker/api/types"
        "github.com/docker/docker/client"
    )
    
    type ContainerInfo struct {
        Name, ID string
    }
    
    func ListRunningContainers() []ContainerInfo {
        cli, err := client.NewEnvClient()
        if err != nil {
            log.Fatal(err)
        }
    
        containers, err := cli.ContainerList(context.Background(), types.ContainerListOptions{})
        if err != nil {
            log.Fatal(err)
        }
    
        container_list := make([]ContainerInfo, 0)
        for _, container := range containers {
            container_list = append(container_list, ContainerInfo{container.Image, container.ID[:10]})
        }
        return container_list
    }
    
    func printContainerList(container_list []ContainerInfo) {
        for _, container := range container_list {
            fmt.Printf("%s, %s\n", container.Name, container.ID)
        }
    }
    
    func deleteFlagFile(alreadyBeingDeletedFlag string, container ContainerInfo) {
    
        err := os.Remove(alreadyBeingDeletedFlag)
        if err != nil {
            log.Fatalf("unable to delete flag file %s for container %s ID=%s", alreadyBeingDeletedFlag,
                container.Name, container.ID)
        }
    }
    func checkIfExistsFlag(alreadyBeingDeletedFlag string, container ContainerInfo) (bool, error) {
    
        found := false
        fi, err := os.Stat(alreadyBeingDeletedFlag)
        if err == nil {
            found = true
            modtime := fi.ModTime()
            duration := time.Since(modtime)
            if duration.Minutes() > 20 {
                log.Infof("looks like file was not deleted in last execution, cleaning up...")
                found = false
                deleteFlagFile(alreadyBeingDeletedFlag, container)
            }
            log.Debugf("container %s already scheduled for being deleted ", container.Name)
            return found, err
        }
    
        return found, err
    
    }
    
    func ScheduleContainerStop(ctx context.Context, container ContainerInfo) {
    
        tmp_prefix_path := ctx.Value("tmp_flags_file_path").(string)
        alreadyBeingDeletedFlag := tmp_prefix_path + container.ID
        flag, err := checkIfExistsFlag(alreadyBeingDeletedFlag, container)
        if flag {
            log.Fatal(err)
        }
        log.Infof("scheduled container %s for stopping", container.Name)
        timeout_duration, err := time.ParseDuration(ctx.Value("exposure_time").(string))
        if err != nil {
            log.Fatalf("incorrect format for exposure_timeout")
        }
        log.Debug("ScheduleContainerStop: outside the lambda function waiting for DONE signal")
    
        //wait for the exposure_time
        timer := time.NewTimer(timeout_duration)
        runtime.Gosched()
        <-timer.C
        var data []byte
        data = make([]byte, 1)
        // creating the flag
        err = ioutil.WriteFile(alreadyBeingDeletedFlag, data, 0644)
        if err != nil {
            log.Fatal(err)
        }
        StopContainer(ctx, container)
        deleteFlagFile(alreadyBeingDeletedFlag, container)
        log.Debug("ScheduleContainerStop: Lambda function DONE")
    
    }
    
    func GetContainerByName(ContainerName string, container_list []ContainerInfo) (ContainerInfo, bool) {
        for _, container := range container_list {
            if container.Name == ContainerName {
                return container, true
            }
        }
        return ContainerInfo{}, false
    }
    
    func GetContainerByID(ContainerID string, container_list []ContainerInfo) (ContainerInfo, bool) {
        if len(ContainerID) <= 0 {
            log.Debug("GetContainerByID: NIL container id provided")
            return ContainerInfo{}, false
        }
    
        if len(container_list) <= 0 {
            log.Debug("GetContainerByID: NIL container_list provided")
            return ContainerInfo{}, false
        }
        for _, container := range container_list {
            if len(container.ID) < len(ContainerID) {
                log.Debugf("incomparable ID, provided ID is larger 
                            than existing one %s %d %s %d", container.ID,
                            len(container.ID), ContainerID, len(ContainerID))
                ContainerID = ContainerID[:len(container.ID)]
            }
            //convert to string
            IDstr := fmt.Sprint(container.ID)
            if strings.HasPrefix(IDstr, ContainerID) {
                return container, true
            }
        }
        return ContainerInfo{}, false
    }
    
    func StopContainer(ctx context.Context, container ContainerInfo) {
        log.Infof("Stopping container %s NOW!", container.Name)
        cli, err := client.NewEnvClient()
        if err != nil {
            log.Fatal(err)
        }
        timeout, err := time.ParseDuration(ctx.Value("container_api_timeout").(string))
        if err != nil {
            log.Fatalf("incorrect format for api timeout")
        }
        err = cli.ContainerStop(context.Background(), container.ID, &timeout)
        if err != nil {
            log.Fatal(err)
        }
    
        log.Infof("container %s has been stopped", container.Name)
    }
    
    
\end{minted}
\captionof{listing}{container\_recycler - handlers/container.go \label{listing:container-recycler-container}}


\begin{minted}[fontsize=\scriptsize]{go}
    package handlers
    
    import (
        "bufio"
        "context"
        "encoding/json"
        log "github.com/Sirupsen/logrus"
        "io"
        "os"
        "regexp"
        "sync"
        "time"
    )
    
    type FalcoNotification struct {
        RawOutput         string    `json:"output"`
        Priority          string    `json:"priority"`
        RuleNameTriggered string    `json:"rule"`
        Time              time.Time `json:"time"`
    }
    
    func SetupLogging() {
        log.SetFormatter(&log.TextFormatter{})
        log.SetOutput(os.Stdout)
        log.SetLevel(log.DebugLevel)
    }
    
    //{"output":"14:54:07.709160152: Alert Shell spawned in a container other than entrypoint 
    // (user=root ssh (id=52d928d8b2a3) shell=sh parent=watch cmdline=sh -c id)",
    // "priority":"Alert","rule":"Run shell in container","time":"2017-03-31T14:54:07.709160152Z"}
    
    func handle(ctx context.Context, f FalcoNotification, wg sync.WaitGroup) {
    
        log.Info(f)
    
        if f.Priority == "Alert" {
    
            var myExp = namedRegexp{regexp.MustCompile(
                `.*\(user=(?P<user>[[:alpha:]]+)\s+(?P<image_name>[[:alpha:]]+)\s+\(id=(?P<image_id>[[:alnum:]]{6,})\).*\)`
            )}
            data := myExp.FindStringSubmatchMap(f.RawOutput)
            log.Debug(data)
    
            log.Debug("Alert received, will try to stop container")
            container_list := ListRunningContainers()
    
            ctx = context.WithValue(ctx, "container_list", container_list)
            container, found := GetContainerByID(data["image_id"], container_list)
            if found {
                log.Debug("FalcoNotification.handle: stopping container")
                ScheduleContainerStop(ctx, container)
            } else {
                log.Warnf("Alert received relative for container ID %s (name=%s) not found in running containers", 
                           data["image_id"], data["image_name"])
            }
        }
    }
    
    func ParseFalcoNotifications(r io.Reader, ctx context.Context) {
        var f FalcoNotification
        scanner := bufio.NewScanner(r)
        var wg sync.WaitGroup
        for scanner.Scan() {
            if err := json.Unmarshal(scanner.Bytes(), &f); err != nil {
                log.Error(err)
    
                log.Debug("ParseFalcoNotifications: Bad FalcoNotification format")
                continue
            }
            wg.Add(1)
            log.Debug("ParseFalcoNotifications: received a falco notification")
    
            go func() {
                defer wg.Done()
                handle(ctx, f, wg)
            }()
        }
        wg.Wait()
    }

\end{minted}
\captionof{listing}{container\_recycler - handlers/falco\_notification.go \label{listing:container-recycler-falco-notification}}

\begin{minted}[fontsize=\scriptsize]{go}
    package handlers
    
    import (
        "regexp"
    )
    
    // embed regexp.Regexp in a new type so we can extend it
    type namedRegexp struct {
        *regexp.Regexp
    }
    
    // add a new method to our new regular expression type
    func (r *namedRegexp) FindStringSubmatchMap(s string) map[string]string {
        captures := make(map[string]string)
    
        match := r.FindStringSubmatch(s)
        if match == nil {
            return captures
        }
    
        for i, name := range r.SubexpNames() {
            // Ignore the whole regexp match and unnamed groups
            if i == 0 || name == "" {
                continue
            }
    
            captures[name] = match[i]
    
        }
        return captures
    }    
\end{minted}
\captionof{listing}{container\_recycler - handlers/namedRegexp.go \label{listing:container-recycler-named-regexp}}

\subsection{\emph{Sinker Registry API}}
\label{subsec:sinker-registry-api-src-code}
Se incluye un extracto del código de \emph{Sinker Registry API} por completitud y como soporte a la lectura de la memoria. Sin
embargo se recomienda su exploración en GitHub en \href{https://github.com/Fsero/sinker\_registry\_api}{https://github.com/Fsero/sinker\_registry\_api} , dónde será más fácil interactuar con el código y ver cambios recientes si los 
hubiera.

\begin{minted}[fontsize=\scriptsize]{go}
    package controllers
    
    import (
        "encoding/json"
    
        "fmt"
    
        "bitbucket.org/fseros/sinker_registry_api/models"
        log "github.com/Sirupsen/logrus"
        "github.com/astaxie/beego"
    )
    
    // Operations about probe
    type ProbeController struct {
        beego.Controller
    }
    
    func (p *ProbeController) URLMapping() {
        p.Mapping("Post", p.Post)
        p.Mapping("GetOne", p.Get)
        p.Mapping("GetAll", p.GetAll)
        p.Mapping("Put", p.Put)
        p.Mapping("Delete", p.Delete)
    }
    
    // @Title Create Probe
    // @Description create new probe
    // @Success 201 {object} models.Probe
    // @Param  fqdn  body string true "fqdn address of the probe"
    // @Param  ipv4  body string true "ipv4 address of the probe"
    // @Param  ipv6  body string false "ipv6 address of the probe"
    // @Param  provider  body string true "cloud provider of the probe"
    // @Param  geolongitude  body float false "geolongitude of the probe"
    // @Param  geolatitude  body float false "geolongitude of the probe"
    // @Param  sshprivatekey  body string false "ssh private key of the probe"
    // @Param  sshpublickey  body string false "ssh public key of the probe"
    // @Param  enabled  body bool false "probe status" "true"
    // @router / [post]
    func (p *ProbeController) Post() {
        var pr models.Probe
        pr.SetDefaults()
        json.Unmarshal(p.Ctx.Input.RequestBody, &pr)
        log.Debugf(" received %v via POST", pr)
        probeid, err := models.AddOne(pr)
        if err != nil {
            p.Data["json"] = fmt.Sprintf("{ 'msg': '%s' }", err.Error())
            p.Ctx.Output.SetStatus(400)
            p.ServeJSON()
            return
        }
        p.Data["json"] = map[string]string{"ProbeId": probeid}
        p.Ctx.Output.SetStatus(201)
        p.ServeJSON()
    }
    
    // @router /:id [get]
    func (p *ProbeController) Get() {
        ProbeID := getIDbyQueryParamOrAsAParam(p)
        if ProbeID != "" {
            ob, err := models.GetByID(ProbeID)
            if err != nil {
                p.Data["json"] = fmt.Sprintf("{ 'msg': '%s' }", err.Error())
            } else {
                p.Data["json"] = ob
            }
        }
        p.ServeJSON()
    }
    
    // @router / [get]
    func (p *ProbeController) GetAll() {
        obs := models.GetAll()
        p.Data["json"] = obs
        p.ServeJSON()
    }
    
    // @router /ip/:ip [get]
    func (p *ProbeController) GetByIP() {
        probeIP := p.Ctx.Input.Param(":ip")
        fmt.Printf("Looking for probes with ip %s", probeIP)
        if probeIP != "" {
            obs, err := models.GetByIPv4(probeIP)
            if err == nil {
                fmt.Printf("Found \n %+v", obs)
                newobs := make([]models.Probe, 0)
                for _, ob := range obs {
                    if ob.Enabled {
                        newobs = append(newobs, ob)
                    }
                }
                fmt.Printf("Found newobs \n %+v", newobs)
                p.Data["json"] = newobs
                p.Ctx.Output.SetStatus(200)
                p.ServeJSON()
            } else {
                p.Data["json"] = fmt.Sprintf("{ 'msg': '%s' }", err.Error())
                p.Ctx.Output.SetStatus(500)
                p.ServeJSON()
                p.StopRun()
            }
        }
    }
    
    // @router /name/:fqdn [get]
    func (p *ProbeController) GetByFQDN() {
        probeName := p.Ctx.Input.Param(":fqdn")
        fmt.Printf("Looking for probes with name %s", probeName)
        if probeName != "" {
            obs, err := models.GetByFQDN(probeName)
            if err == nil {
                fmt.Printf("Found \n %+v", obs)
                newobs := make([]models.Probe, 0)
                for _, ob := range obs {
                    if ob.Enabled {
                        newobs = append(newobs, ob)
                    }
                }
                fmt.Printf("Found newobs \n %+v", newobs)
                p.Data["json"] = newobs
                p.Ctx.Output.SetStatus(200)
                p.ServeJSON()
            } else {
                p.Data["json"] = fmt.Sprintf("{ 'msg': '%s' }", err.Error())
                p.Ctx.Output.SetStatus(500)
                p.ServeJSON()
                p.StopRun()
            }
        }
    }
    
    // @router /disable/?:id [put]
    func (p *ProbeController) Disable() {
        ProbeID := getIDbyQueryParamOrAsAParam(p)
        if ProbeID != "" {
            log.Infof("[controllers.probe.Disable]: disabling probe %s", ProbeID)
            ob, err := models.Disable(ProbeID)
            if err != nil {
                p.Data["json"] = fmt.Sprintf("{ 'msg': '%s' }", err.Error())
            } else {
                p.Data["json"] = ob
            }
        }
        p.ServeJSON()
    }
    
    func getIDbyQueryParamOrAsAParam(p *ProbeController) string {
        var ProbeID string
        ProbeID = p.GetString("id")
        if ProbeID == "" {
            ProbeID = p.Ctx.Input.Param(":id")
        }
        return ProbeID
    
    }
    
    // @router /enable/?:id [put]
    func (p *ProbeController) Enable() {
        ProbeID := getIDbyQueryParamOrAsAParam(p)
        if ProbeID != "" {
            log.Infof("[controllers.probe.Enable]: enabling probe %s", ProbeID)
            ob, err := models.Enable(ProbeID)
            if err != nil {
                p.Data["json"] = fmt.Sprintf("{ 'msg': '%s' }", err.Error())
            } else {
                p.Data["json"] = ob
            }
        }
        p.ServeJSON()
    }
    
    // @Title Updates traces path
    // @Description updates traces path
    // @Success 200 {object} models.Probe
    // @Param  tracespath  body string false "traces path for probe"
    // @router /tracespath/?:id [put]
    func (p *ProbeController) UpdateTracesPath() {
        ProbeID := getIDbyQueryParamOrAsAParam(p)
        if ProbeID != "" {
            log.Infof("[controllers.probe.UpdateTracesPath]: updating traces path for probe %s", ProbeID)
            var pr models.Probe
            json.Unmarshal(p.Ctx.Input.RequestBody, &pr)
            ob, err := models.UpdateTracesPath(ProbeID, pr.TracesPath)
            if err != nil {
                p.Data["json"] = fmt.Sprintf("{ 'msg': '%s' }", err.Error())
            } else {
                p.Data["json"] = ob
            }
        }
        p.ServeJSON()
    }
    
    // @Title Updates ssh key
    // @Description updates ssh key
    // @Success 200 {object} models.Probe
    // @Param  sshprivatekey  body string false "ssh private key of the probe"
    // @Param  sshpublickey  body string false "ssh public key of the probe"
    // @router /ssh/?:id [put]
    func (p *ProbeController) UploadSSH() {
        ProbeID := getIDbyQueryParamOrAsAParam(p)
        if ProbeID != "" {
            log.Infof("[controllers.probe.UploadSSH]: updating ssh key for probe %s", ProbeID)
            var pr models.Probe
            json.Unmarshal(p.Ctx.Input.RequestBody, &pr)
            ob, err := models.UploadSSH(ProbeID, pr.SSHPrivateKey, pr.SSHPublicKey)
            if err != nil {
                p.Data["json"] = fmt.Sprintf("{ 'msg': '%s' }", err.Error())
            } else {
                p.Data["json"] = ob
            }
        }
        p.ServeJSON()
    }
    
    // @Title Gets ssh key
    // @Description updates ssh key
    // @Success 200 {object} models.Probe
    // @Param  sshprivatekey  body string false "ssh private key of the probe"
    // @Param  sshpublickey  body string false "ssh public key of the probe"
    // @router /ssh/?:id [get]
    func (p *ProbeController) GetSSH() {
        ProbeID := getIDbyQueryParamOrAsAParam(p)
        if ProbeID != "" {
            log.Infof("[controllers.probe.GetSSH]: Getting ssh key for probe %s", ProbeID)
            keys, err := models.GetSSH(ProbeID)
            if err != nil {
                p.Data["json"] = fmt.Sprintf("{ 'msg': '%s' }", err.Error())
            } else {
                p.Data["json"] = map[string]string{"ProbeId": ProbeID, 
                "SSHPrivateKey": keys.Private, 
                "SSHPublicKey": keys.Public}
    
            }
        }
        p.ServeJSON()
    }
    
    func (p *ProbeController) Put() {
        var pr models.Probe
        pr.SetDefaults()
    
        json.Unmarshal(p.Ctx.Input.RequestBody, &pr)
        log.Debugf(" received %v via POST", pr)
        probeid, err := models.AddOne(pr)
        if err != nil {
            p.Data["json"] = fmt.Sprintf("{ 'msg': '%s' }", err.Error())
            p.Ctx.Output.SetStatus(400)
            p.ServeJSON()
            return
        }
    
        p.Data["json"] = map[string]string{"ProbeId": probeid}
        p.Ctx.Output.SetStatus(201)
        p.ServeJSON()
    }
    
    // @router /delete/?:id [delete]
    func (p *ProbeController) Delete() {
        ProbeID := getIDbyQueryParamOrAsAParam(p)
        if ProbeID != "" {
            log.Infof("[controllers.probe.Delete]: deleting probe %s", ProbeID)
            _, err := models.Delete(ProbeID)
            if err != nil {
                p.Data["json"] = fmt.Sprintf("{ 'msg': '%s' }", err.Error())
            } else {
                p.Data["json"] = fmt.Sprintf("{ 'msg': 'deleted probe with id %s' }", ProbeID)
            }
        }
        p.ServeJSON()
    }
    
\end{minted}
\captionof{listing}{sinkerregistryapi - controllers/probe.go \label{listing:sinker-registry-api-controllers-probe}}

\subsection{\emph{metadata\_extractor}}
\label{subsec:metadata-extractor-src-code}
Se incluye un extracto del código de \emph{metadata\_extractor} por completitud y como soporte a la lectura de la memoria. Sin
embargo se recomienda su exploración en GitHub en \href{https://github.com/metadata\_extractor}{metadata\_extractor} , dónde será más fácil interactuar con el código y ver cambios recientes si los 
hubiera.

\begin{minted}[fontsize=\scriptsize,linenos]{go}
    package parsers
    
    import (
        "bufio"
        "bytes"
        "encoding/json"
        "fmt"
        "os/exec"
        "regexp"
        "sort"
        "strconv"
        "strings"
    
        "bitbucket.org/fseros/metadata_extractor/helpers"
        "github.com/Sirupsen/logrus"
    )
    
    var loginAttemptRegexp = regexp.MustCompile(`(res=\d+) (data=.*PAM:authentication.*)(acct=.*)(exe=.*)
                          (hostname=.*)(addr=[\d{1,3}\.]+).*(res=failed|success).*`)
    var passwordInputRegexp = regexp.MustCompile(`(res=\d+) (data=.\w+.*)`)
    
    func Init() bool {
        _, err := exec.Command("/usr/bin/sysdig", "-h").Output()
        if err != nil {
            logrus.Fatalf("Unable to initialize ssh parsers %s", err)
            return false
        }
        return true
    }
    
    func extractLoginAttempt(fields []string) AttackerLoginAttempt {
    
        var capture AttackerLoginAttempt
        var subfields []string
        subfields = make([]string, 1)
    
        for key, field := range fields {
            if key == 0 {
                continue
            }
            if key == 2 {
                subfields = append(subfields, fmt.Sprintf("%s", field))
                continue
            }
            helpers.SplitFieldsBySep("=", field, &subfields)
        }
        capture.User = subfields[3]
        capture.Hostname = subfields[5]
        capture.IP = subfields[6]
        capture.Successful = subfields[7] == "success"
        return capture
    }
    
    func extractPassword(fields []string) AttackerLoginAttempt {
        var capture AttackerLoginAttempt
        var subfields []string
        subfields = make([]string, 1)
    
        for key, field := range fields {
            if key == 0 {
                continue
            }
            helpers.SplitFieldsBySep("=", field, &subfields)
        }
        str := strings.Join(subfields[2:], "")
        if strings.Contains(str, "PAM:") || len(str) > 30 {
            capture.Password = "'NOTFOUND'"
        } else {
            capture.Password = subfields[2]
        }
        return capture
    }
    
    func matchesRegexpTrace(trace Trace, re *regexp.Regexp) bool {
    
        str := strings.Replace(trace.EventInfo, "\n", "", -1)
        return re.MatchString(str)
    
    }
    
    func parseTraces(traces []Trace) []AttackerLoginAttempt {
        // on traces password appears first and then login attempt info.
        // traces should be ordered by time
        /*
            {63f6e3883d7c ssh 0 < res=10 data=
            abc123  159003 1496213704714312629 write sshd 21037 13946}
            {63f6e3883d7c ssh 0 < res=140 data=
            Lop=PAM:authentication acct="root" exe="/usr/sbin/sshd" hostname=107.160.16.221 
            addr=107.160.16.221 terminal=ssh res=success  159104 1496213704727977064 
            sendto sshd 21036 13945}
    
        */
        var m map[int][]Trace
        m = make(map[int][]Trace, 0)
        for _, trace := range traces {
            logrus.Debugf("[parseTraces] readed %+v", trace)
            tr := m[trace.ThreadTid]
            if tr == nil {
                tr = make([]Trace, 0)
            }
            tr = append(tr, trace)
            m[trace.ThreadTid] = tr
        }
    
        var keys []int
        for k := range m {
            keys = append(keys, k)
        }
        sort.Ints(keys)
        var LoginAttempts []AttackerLoginAttempt
        LoginAttempts = make([]AttackerLoginAttempt, 0)
    
        for k := 0; k < len(keys); k += 2 {
            var trace0, trace1 []Trace
            trace0 = m[keys[k]]
            logrus.Debugf("%d %+v\n", keys[k], m[keys[k]])
            if k+1 < len(keys) {
                trace1 = m[keys[k+1]]
                logrus.Debugf("%d %+v\n", keys[k+1], m[keys[k+1]])
            }
            switch {
            case len(trace0) <= 0 && len(trace1) <= 0:
                logrus.Fatalf("[parseTraces] invalid block, something nasty happened")
    
            case len(trace0) <= 0 && len(trace1) > 0:
                trace0, trace1 = trace1, trace0
                logrus.Debugf("[parseTraces] swapping blocks, something weird happened")
            }
    
            for i := range trace0 {
                var elem0, elem1 Trace
                if i < len(trace0) {
                    elem0 = trace0[i]
                } else {
                    elem0 = Trace{}
                }
                if i < len(trace1) {
                    elem1 = trace1[i]
                } else {
                    elem1 = Trace{}
                }
                var capture, p, l AttackerLoginAttempt
                logrus.Debugf("elem0 %+v elem1 %+v", elem0, elem1)
                switch {
                case matchesRegexpTrace(elem0, passwordInputRegexp) && matchesRegexpTrace(elem1, loginAttemptRegexp):
    
                    str := strings.Replace(elem0.EventInfo, "\n", "", -1)
                    fields := passwordInputRegexp.FindStringSubmatch(str)
                    p = extractPassword(fields)
                    str = strings.Replace(elem1.EventInfo, "\n", "", -1)
                    fields = loginAttemptRegexp.FindStringSubmatch(str)
                    l = extractLoginAttempt(fields)
                    l.UnixTime = (strconv.FormatInt(elem1.EventOutputUnixTime, 10)[0:13])
                    l.ContainerID = elem1.ContainerID
                    logrus.Debug("elem0 == pass and elem1 == login")
    
                case matchesRegexpTrace(elem1, passwordInputRegexp) && matchesRegexpTrace(elem0, loginAttemptRegexp):
                    str := strings.Replace(elem1.EventInfo, "\n", "", -1)
                    fields := passwordInputRegexp.FindStringSubmatch(str)
                    p = extractPassword(fields)
                    str = strings.Replace(elem0.EventInfo, "\n", "", -1)
                    fields = loginAttemptRegexp.FindStringSubmatch(str)
                    l = extractLoginAttempt(fields)
                    l.UnixTime = (strconv.FormatInt(elem0.EventOutputUnixTime, 10)[0:13])
                    l.ContainerID = elem0.ContainerID
                    logrus.Debug("elem1 == pass and elem0 == login")
    
                case matchesRegexpTrace(elem0, passwordInputRegexp) && matchesRegexpTrace(elem1, passwordInputRegexp):
                    logrus.Debugf("[parseTraces] two password blocks, Discarding")
                    continue
    
                case matchesRegexpTrace(elem1, loginAttemptRegexp) && matchesRegexpTrace(elem0, loginAttemptRegexp):
                    logrus.Debugf("[parseTraces] two login blocks, Discarding")
                    continue
    
                default:
                    if elem0.ContainerID == "" {
                        switch {
                        case matchesRegexpTrace(elem1, loginAttemptRegexp):
                            str := strings.Replace(elem1.EventInfo, "\n", "", -1)
                            fields := loginAttemptRegexp.FindStringSubmatch(str)
                            l = extractLoginAttempt(fields)
                            l.UnixTime = (strconv.FormatInt(elem1.EventOutputUnixTime, 10)[0:13])
                            p.Password = `'NOTFOUND'`
                            l.ContainerID = elem1.ContainerID
                            logrus.Debugf("[parseTraces] no password found :-(")
                            logrus.Debug("elem0 == [] and elem1 == login")
                        }
                    } else if elem1.ContainerID == "" {
                        switch {
                        case matchesRegexpTrace(elem0, loginAttemptRegexp):
                            str := strings.Replace(elem0.EventInfo, "\n", "", -1)
                            fields := loginAttemptRegexp.FindStringSubmatch(str)
                            l = extractLoginAttempt(fields)
                            l.UnixTime = (strconv.FormatInt(elem0.EventOutputUnixTime, 10)[0:13])
                            p.Password = `'NOTFOUND'`
                            l.ContainerID = elem0.ContainerID
                            logrus.Debugf("[parseTraces] no password found :-(")
                            logrus.Debug("elem0 == login and elem1 == [] ")
                        }
                    } else {
                        logrus.Fatalf("[parseTraces] unexpected error, 
                        we didnt receive any block %+v %+v", trace0, trace1)
                    }
    
                }
                capture.Password = p.Password
                capture.ContainerID = l.ContainerID
                capture.Hostname = l.Hostname
                capture.IP = l.IP
                capture.Successful = l.Successful
                capture.UnixTime = l.UnixTime
                capture.User = l.User
    
                if validateCapture(capture) {
                    logrus.Debugf("[parseTraces] adding capture %+v", capture)
                    LoginAttempts = append(LoginAttempts, capture)
                }
    
            }
        }
        return LoginAttempts
    }
    
    func ExtractAttackerLoginAttempt(file string) []AttackerLoginAttempt {
        //sysdig -j -A -F -r srv02.superprivyhosting.com.2017-05-31-06-54.part2 
        // container.id!=host and fd.num=4 and evt.is_io_write=true 
        // and evt.dir = '<' and proc.name=sshd | egrep -B1 PAM:
    
        sysdig := exec.Command("/usr/bin/sysdig", "-pc", "-j", "-F", "-A", "-r", file, 
                  "container.id!=host", "and", "fd.num=4", "and", 
                  "evt.is_io_write=true", "and", "evt.dir", 
                  "=", "'<'", "and", "proc.name=sshd")
        egrep := exec.Command("egrep", "-B1", "PAM:")
        removedashes := exec.Command("egrep", "-v", "\\-")
        output, stderr, err := helpers.Pipeline(sysdig, egrep, removedashes)
        logrus.Debug(sysdig)
        logrus.Debug(egrep)
        logrus.Debugf("STDERR: %s", stderr)
        if err != nil {
            if checkSysdigFailure(output, stderr, file) {
                logrus.Debugf("[ExtractAttackerActivity] Unable to launch sysdig %s", err)
                return nil
            }
        }
        var traces []Trace
    
        scanner := bufio.NewScanner(bytes.NewReader(output))
        for scanner.Scan() {
            var tr Trace
            line := scanner.Text()
            if len(line) == 0 {
                continue
            }
            logrus.Debugf("[ExtractAttackerLoginAttempt] readed \n %s", line)
            if err := json.Unmarshal([]byte(line), &tr); err != nil {
                logrus.Debugf("[ExtractAttackerLoginAttempt] Unable to get JSON %s ", err)
                logrus.Debugf("[ExtractAttackerLoginAttempt] Unable to parse trace from %s", line)
                continue
            }
            tr.EventInfo = strings.Replace(tr.EventInfo, "\n", "", -1)
            switch {
            case !loginAttemptRegexp.MatchString(tr.EventInfo) && !passwordInputRegexp.MatchString(tr.EventInfo):
                logrus.Debugf("[extractAttackerLoginAttempt] invalid trace, discarding it")
                continue
            case strings.Contains(tr.EventInfo, "ssh:notty"):
                logrus.Debugf("[extractAttackerLoginAttempt] invalid trace, discarding it")
                continue
            case loginAttemptRegexp.MatchString(tr.EventInfo) && !strings.Contains(tr.EventInfo, "PAM:authentication"):
                logrus.Debugf("[extractAttackerLoginAttempt] we are only interested on auth events from PAM")
                continue
            default:
                traces = append(traces, tr)
            }
    
        }
        if err := scanner.Err(); err != nil {
            logrus.Fatalf("[ExtractAttackerLoginAttempt] Unable to parse trace %s", err)
        }
        logrus.Debugf("num of traces %s", len(traces))
        sort.Sort(ByEventNumber(traces))
        LoginAttempts := parseTraces(traces)
        return LoginAttempts
    }
    
    //it reads raw data from command and outputs
    //a formatted list of lines
    func parseActivities(lines []byte) []string {
        headerRegexp := regexp.MustCompile(`(\d+) (\d{4}-\d{2}-\d{2} \d{2}:\d{2}:\d{2}\.\d{9}) (\w+)@(\w+)\)`)
        scanner := bufio.NewScanner(bytes.NewReader(lines))
        var buffer bytes.Buffer
        AttackerActivityLog := make([]string, 0)
    
        for scanner.Scan() {
            line := scanner.Text()
            if len(line) == 0 {
                continue
            }
            if headerRegexp.MatchString(line) {
                if buffer.Len() > 0 {
                    AttackerActivityLog = append(AttackerActivityLog, buffer.String())
                    buffer.Reset()
                }
                buffer.WriteString(line)
            } else {
                buffer.WriteString(line)
            }
    
        }
        return AttackerActivityLog
    }
    
    func isTraceFileOk(output, file string) bool {
        var isOk bool
        if strings.Contains(output, "Is the file truncated?") {
            isOk = false
            logrus.Debugf("looks like file is not complete, refusing to continue")
        } else if ( strings.Contains(output, "error reading from file") 
               || strings.Contains(output, "unexpected end of file") ) {
            logrus.Debugf("looks like file is wrong, refusing to continue")
            isOk = false
        } else {
            logrus.Debugf("looks like file is fine, why are we here?")
            isOk = true
        }
        return isOk
    }
    
    func checkSysdigFailure(stdout, stderr []byte, file string) bool {
        sErr := string(stderr[:])
        sOut := string(stdout[:])
        if !isTraceFileOk(sErr, file) {
            logrus.Debugf("STDERR: %s", sErr)
            logrus.Debugf("[checkSysdigFailure] unable to read trace %s refusing to continue", file)
            return true
        } else if !isTraceFileOk(sOut, file) {
            logrus.Debugf("STDOUT: %s", sOut)
            logrus.Debugf("[checkSysdigFailure] unable to read trace %s refusing to continue", file)
            return true
        }
        return false
    }
    
    func ExtractAttackerActivity(file string) []AttackerActivity {
    
        //sysdig -pc -r test/srv02.superprivyhosting.com.2017-05-31-06-54.part2 -c spy_users
        // '100 disable_color' container.id != host | grep -v 'sshd -R'
        sysdig := exec.Command("/usr/bin/sysdig", "-pc", "-c", "spy_users", "-r", file, "container.id!=host")
        egrep := exec.Command("egrep", "-v", `sshd.*R`)
    
        output, stderr, err := helpers.Pipeline(sysdig, egrep)
        if err != nil {
            if checkSysdigFailure(output, stderr, file) {
                logrus.Debugf("[ExtractAttackerActivity] Unable to launch sysdig %s", err)
                return nil
    
            }
            // if stderr is not empty, then something nasty happened if not is just an empty file
        }
        AttackerActivityLog := parseActivities(output)
        AttackerActivityEntries := make([]AttackerActivity, 0)
        ActivityRegexp := regexp.MustCompile(`(\d+) (\d{4}-\d{2}-\d{2} 
        \d{2}:\d{2}:\d{2}\.\d{9}) (\w+)@(\w+)\) (.*)`)
        for _, logentry := range AttackerActivityLog {
            subfields := ActivityRegexp.FindStringSubmatch(logentry)
            if len(subfields) == 0 {
                continue
            }
            var entry activitylog
            entry.PID = subfields[1]
            entry.User = subfields[3]
            entry.Datetime = subfields[2]
            entry.ContainerID = subfields[4]
            entry.Command = subfields[5]
            logrus.Debugf("entry parsed %+v", entry)
            if validateEntry(entry) {
                logrus.Debugf("entry validated")
                var data AttackerActivity
                data.Activity = entry.Command
                data.ContainerID = entry.ContainerID
                data.SourceFile = file
                data.PID = entry.PID
                data.User = entry.User
                data.Datetime = entry.Datetime
                AttackerActivityEntries = append(AttackerActivityEntries, data)
    
            }
    
        }
        return AttackerActivityEntries
    }
    
    
\end{minted}
\captionof{listing}{metadata\_extractor - handlers/ssh.go \label{listing:metadata-extractor-handlers-ssh}}

%%% Local Variables: 
%%% mode: latex
%%% TeX-master: "main"
%%% End: