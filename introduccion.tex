
\chapter{Prólogo}

La digitalización de la sociedad no es una idea de futuro. Hace años que funciones críticas para la sociedad
como son las tareas ejercidas en hospitales, entidades financieras o centrales nucleares son gestionadas mediante sistemas informáticos. 
La ubicuidad de dispositivos, ordenadores, móviles, dispositivos electrónicos del \emph{internet of things}
(electrodomésticos, montajes con Arduino o Raspberry Pi...) unido al afianzamiento de la informatización en el ámbito empresarial, ha generado
que dichos sistemas y dispositivos se consideren activos a proteger de atacantes de índole privada o gubernamental.

La seguridad informática ha pasado de ser un área de interés para dichas corporaciones a ser considerada de interés general.
Hasta no hac tanto tiempo, hubiera sido imposible que una noticia sobre un \emph{Ransomware} apareciese en prensa o televisión no especializada y, sin embargo, esto ha ocurrido
al menos una docena de veces este 2017.

El sector de la seguridad informática es especialmente interesante desde el punto de vista del defensor, puesto que
los sistemas de detención y prevención han de cotrarrestar con talento y recursos limitados los ataques de adversarios
que poseen muchos más recursos, talento y motivación para triunfar.

Es por ello que se debe asumir que -en dicho escenario- la derrota, es decir la intrusión, tiene un alto factor de éxito y es 
por ello que la gestión de riesgos no se basa en la hipótesis de si el ataque será exitoso o no, sino en que, cuando
este ataque suceda, se aprenda lo máximo posible para evitarlo en otra ocasión y limitar sus efectos.

Así pues, este proyecto se enmarca en el campo de la contrainteligencia y en él se describe un método que permite al defensor conocer cómo algunos atacantes intentan vulnerar sus sistemas. 






%%% Local Variables: 
%%% mode: latex
%%% TeX-master: "index"
%%% End: 
