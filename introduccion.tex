
\chapter{Introducción}

La digitalización de la sociedad no es una idea de futuro, hace años que funciones críticas para la sociedad
desde hospitales, entidades financieras ó centrales nucleares son gestionados mediante sistemas informáticos. 
La ubicuidad de dispositivos, ordenadores, móviles, dispositivos electronicos del \emph{internet of things}
( electrodomesticos, montajes con Arduino o Raspberry Pi ... ) unido al afianzamiento de la informatizacion en el ámbito empresarial, ha generado
que dichos sistemas y dispositivos se consideren activos a proteger de atacantes de indole privado o gobernamental.

La seguridad informática ha pasado de ser un area de interes para dichas corporaciones a ser considerada de interes general,
hubiera sido imposible que una noticia sobre un \emph{Ransomware} apareciese en prensa no especializada ó television, sin embargo esto ha ocurrido
al menos una docena de veces éste 2017.

El sector de la seguridad informática es especialmente interesante desde el punto de vista del defensor, puesto que
los sistemas de detención y prevención han de cotrarrestar con talento y recursos limitados los ataques de adversarios
que possen mucho más recursos, talento y motivación para triunfar.

Es por ello que se debe de asumir que en dicho escenario la derrota, es decir la intrusión, tiene un alto factor de éxito es 
por ello que la gestión de riesgos no se basa sobre la hipotesis de si el ataque será exitoso o no, sino en que cuando
éste ataque suceda se aprenda lo maximo posible para evitarlo en otra ocasión y limitar sus efectos.

Este proyecto se enmarca en el campo de la contrainteligencia, y en el se describe un metodo que permite al defensor conocer como algunos atacantes intentan vulnerar sus sistemas. 






%%% Local Variables: 
%%% mode: latex
%%% TeX-master: "index"
%%% End: 
